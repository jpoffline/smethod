\documentclass[amsmath,amssymb,12pt, eqsecnum]{revtex4}
\usepackage{graphicx, amsmath, amsmath, amssymb}
\usepackage[breaklinks, colorlinks, citecolor=blue]{hyperref}

%\newcommand{\vphi}[0]{{\color{green}{\delta\phi}}}
%\newcommand{\dlg}[0]{{\color{red}{\lp g}}}
\newcommand{\vphi}[0]{\delta\phi}
\newcommand{\lpa}[0]{\lp A}
\newcommand{\dlg}[0]{\lp g}

\newcommand{\depg}[0]{\ep g}
\newcommand{\virt}[0]{\hat{\delta}}
\newcommand{\Dvphi}[1]{\nabla_{#1}\vphi}
\newcommand{\DDvphi}[2]{\nabla_{#1}\nabla_{#2}\vphi}

\newcommand{\Dvvphi}[1]{\nabla_{#1}\virt\vphi}
\newcommand{\DDvvphi}[2]{\nabla_{#1}\nabla_{#2}\virt\vphi}

\newcommand{\tmat}[4]{\left( \begin{array}{cc} #1 & #2 \\ #3 & #4 \end{array}\right)}
\newcommand{\expec}[1]{\left\langle #1\right\rangle}
\newcommand{\half}[0]{\frac{1}{2}}
\newcommand{\cs}[3]{\Gamma^{#1}_{\,\,\,\, #2#3}}

\newcommand{\pd}[2]{\frac{\partial #1}{\partial #2}}
\newcommand{\ld}[0]{\mathcal{L}}
\newcommand{\md}[0]{\mathcal{M}}
\newcommand{\dd}[0]{\textrm{d}}
\newcommand{\defn}[0]{\equiv}
\newcommand{\diag}[0]{\textrm{diag}}
\newcommand{\qsubrm}[2]{{#1}_{\scriptsize{\textrm{#2}}}}
\newcommand{\qsuprm}[2]{{#1}^{\textrm{#2}}}

 \newcommand{\subsm}[2]{{#1}_{\scriptscriptstyle{#2}}}

\newcommand{\supsm}[2]{{#1}^{\scriptscriptstyle{#2}}}
\newcommand{\symmb}[0]{\varrho}
\newcommand{\AW}[0]{A_{\mathcal{W}}}
\newcommand{\BW}[0]{B_{\mathcal{W}}}
\newcommand{\CW}[0]{C_{\mathcal{W}}}
\newcommand{\DW}[0]{D_{\mathcal{W}}}
\newcommand{\EW}[0]{E_{\mathcal{W}}}

\newcommand{\AP}[0]{A_{\mathcal{P}}}
\newcommand{\BP}[0]{B_{\mathcal{P}}}
\newcommand{\CP}[0]{C_{\mathcal{P}}}
\newcommand{\DP}[0]{D_{\mathcal{P}}}
\newcommand{\EP}[0]{E_{\mathcal{P}}}
\newcommand{\FP}[0]{F_{\mathcal{P}}}
\newcommand{\GP}[0]{G_{\mathcal{P}}}

\newcommand{\sol}[0]{\ld_{\scriptscriptstyle\{2\}}}

\newcommand{\tis}[0]{ {\theta}^{\scriptscriptstyle\rm{S}}}
\newcommand{\tisdot}[0]{ {\dot{\theta}}^{\scriptscriptstyle\rm{S}}}
\newcommand{\pis}[0]{ {\Pi}^{\scriptscriptstyle\rm{S}}}
\newcommand{\xis}[0]{ {\xi}^{\scriptscriptstyle\rm{S}}}
\newcommand{\xisdot}[0]{ {\dot{\xi}}^{\scriptscriptstyle\rm{S}}}
\newcommand{\xisddot}[0]{ {\ddot{\xi}}^{\scriptscriptstyle\rm{S}}}
\newcommand{\xisdddot}[0]{ {\dddot{\xi}}^{\scriptscriptstyle\rm{S}}}
\newcommand{\lcdm}[0]{$\Lambda$CDM}

\newcommand{\coup}[0]{\mathcal{Q}}
\newcommand{\vmkin}[0]{\mathcal{K}}



\newcommand{\newchap}[2]{\chapter{#1}
\markboth{\MakeUppercase{Chapter \thechapter.\ #2}}{}}
\newcommand{\nphiu}[1]{\nabla^{#1}\phi}
\newcommand{\nphid}[1]{\nabla_{#1}\phi}

\newcommand{\gbm}[1]{\bm{#1}}
\newcommand{\rbm}[1]{{\bf{#1}}}
\newcommand{\ci}[0]{\textrm{i}}
%\newcolumntype{V}{>{\centering\arraybackslash} m{.4\linewidth} }
\newcommand{\kin}[0]{{\mathcal{X}}}
\newcommand{\hct}[0]{\mathcal{H}}
\renewcommand{\figurename}{Figure}
\newcommand{\ep}[0]{{ {\delta}_{\scriptscriptstyle{\rm{E}}}}}
\newcommand{\lp}[0]{{ {\delta}_{\scriptscriptstyle{\rm{L}}}}}
\def\be{\begin{equation}}
\def\ee{\end{equation}}
\def\bea{\begin{eqnarray}}
\def\eea{\end{eqnarray}}
\def\bse{\begin{subequations}}
\def\ese{\end{subequations}}
\newcommand{\lied}[1]{\pounds_{#1}}

\newcommand{\sech}[0]{\textrm{ sech}}
\newcommand{\fref}[1]{{Figure \ref{#1}}}
\newcommand{\tref}[1]{{Table \ref{#1}}}
\newcommand{\secref}[1]{{section \ref{#1}}}
\newcommand{\Secref}[1]{{Section \ref{#1}}}
% PUT CATCH ON THE END OF SQUARE-ROOT SYMBOLS
\newcommand{\rsbb}[2]{#1_{\mathbb{#2}}}

\newcommand{\dg}[0]{\delta g}
\let\oldsqrt\sqrt
% it defines the new \sqrt in terms of the old one
\def\sqrt{\mathpalette\DHLhksqrt}
\def\DHLhksqrt#1#2{%
\setbox0=\hbox{$#1\oldsqrt{#2\,}$}\dimen0=\ht0
\advance\dimen0-0.2\ht0
\setbox2=\hbox{\vrule height\ht0 depth -\dimen0}%
{\box0\lower0.4pt\box2}}

\newcommand{\sbm}[2]{#1_{\mathbb{#2}}}

\linespread{1}

\begin{document}
\title{Numerics}
\author{Jonathan A. Pearson}
\email{jonathan.pearson@durham.ac.uk}
\affiliation{Centre for Particle Theory, Department of Mathematical Sciences, Durham University, South Road, Durham, DH1 3LE, U.K.}
\date{\today}
\begin{abstract}
 
\end{abstract}

\maketitle
\tableofcontents
\section{Introduction}

These notes contain descriptions of the numerics and algorithms.

The idea is to construct a code to evolve
\bea
\square\Phi + m^2\Phi =0,
\eea
where
\bea
\square\Phi =g^{\mu\nu} \nabla_{\mu}\nabla_{\nu}\Phi  ,
\eea
in which the metric is
\bea
g_{\mu\nu}\dd x^{\mu} \dd x^{\nu} = a^2(\tau)\bigg[ -(1+2V)\dd\tau^2 + (1-2V)\dd x_i\dd x^i \bigg],
\eea
and $V$ solves the Poisson equation,
\bea
\nabla^2 V =   {4\pi G}{ }a^2\qsubrm{\rho}{dust} \bigg( |\Phi|^2 -1\bigg),\qquad \qsubrm{\rho}{dust} = \qsubrm{\bar{\rho}}{dust}/a^3
\eea

\subsection{Literature review}
\cite{1993ApJ...416L..71W}, \cite{Widrow:1996eq}, \cite{Uhlemann:2014npa} 
\subsection{Discretization onto a lattice}
\label{sec:fdd}
Space will be discretized onto a lattice; the spacing between lattice sites is $h$.  A quantity, $Q$, in 2D is discretized so that it lives on the lattice; $Q(x,y) \rightarrow Q_{i,j}$. 


The discretization of the  first derivative of a  function $f(\rbm{x})$ at lattice site ``$i$'' is
\bea
\pd{f}{x} = \frac{f_{i+1} - f_{i-1}}{2h},
\eea
and the second derivative   discretized to second order is
\bea
\pd{^2f}{x^2} = \frac{f_{i+1} + f_{i-1} - 2f_i}{h^2}.
\eea
We could also discretize second derivatives to fourth order via
\bea
\pd{^2f}{x^2} = \frac{-f_{i+2}+ 16f_{i+1} - 30 f_i + 16f_{i-1} -f_{i-2}}{12h^2}.
\eea

\section{Elliptic PDEs}
Here we outline some methods of numerically solving elliptic partial differential equations. As an example, we will show how to solve the 2D Poisson equation,
\bea
\nabla^2V = -\rho.
\eea
Here, $V = V(\rbm{x})$ is a ``potential'', and $\rho = \rho(\rbm{x})$ a source.
Using the finite difference discretization scheme outlined  in section \ref{sec:fdd}, the discrete version of this is
\bea
V_{i+1,j}+V_{i-1,j}-4V_{i,j} + V_{i,j+1}+V_{i,j-1} = -h^2\rho_{i,j}.
\eea
Here we used second order accurate derivative discretization. This can be rearranged to find
\bea
V_{i,j} = \frac{1}{4}\bigg[V_{i+1,j}+V_{i-1,j}+ V_{i,j+1}+V_{i,j-1} + h^2\rho_{i,j}\bigg].
\eea
There is clearly an issue: the value of the potential (which is to be found) relies on knowledge of the potential on its four neighbouring lattice sites. The remedy is to update the potential using (fictitious) time-steps;
\bea
\label{eq:sec:simple-update}
V^{n+1}_{i,j} = \frac{1}{4}\bigg[V^n_{i+1,j}+V^n_{i-1,j}+ V^n_{i,j+1}+V^n_{i,j-1} + h^2\rho_{i,j}\bigg].
\eea
This is Jacobi's iterative method.
As $n$ increments, the values of the potential $V$ converge onto those which solve the Poisson equation. There are clever ways of doing this, and each has different convergence properties.
\subsection{Gauss-Seidel method}
The Jacobi method   (\ref{eq:sec:simple-update}) can be made faster by using the newly updated field values on the RHS,
\bea
V_{i,j}^{n+1} = \frac{1}{4}\bigg[ V_{i+1,j}^n + V^{n+1}_{i-1,j} + V^n_{i,j+1} + V^{n+1}_{i,j-1}+h^2\rho_{i,j} \bigg].
\eea
\subsection{Successive over relaxation}
Another method is to using a different linear combination,
\bea
V_{i,j}^{n+1} = (1-\omega) V^n_{i,j} + \frac{\omega}{4}\bigg[ V_{i+1,j}^n + V^{n+1}_{i-1,j} + V^n_{i,j+1} + V^{n+1}_{i,j-1}+h^2\rho_{i,j} \bigg]
\eea
The parameter $\omega$ is the SoR parameter. Its value determines the convergence of the algorithm:
\begin{itemize}
\item Only convergent if $0<\omega <2$,
\item Faster than Gauss-Seidel if $1 <\omega <2$,
\item Fastest on square lattice if $\omega \sim 2/(1+\pi/L)$, where $L$ is the number of lattice points in each direction.
\end{itemize}
\subsection{Checks against known analytic solutions}
We want to check the convergence properties of the codes against known solutions.
\bea
\rho(x,y) = 2x^3 + 6xy(y-1),\qquad V(x,y) = x^3y(1-y)
\eea
\section{Hyperbolic PDEs}
Here we outline how to solve hyperbolic PDEs; that is, wave equations of the form
\bea
\ddot{\Phi} - \nabla^2\Phi + U'(\Phi) = S(\rbm{x}).
\eea
Here, $S(\rbm{x})$ is some source term (which will be important for the applications we have in mind), and  $U(\Phi)$ is the scalar field potential; typically, $U = \half m^2\Phi^2$ is a mass term. First, we trivially re-write this as
\bea
\ddot{\Phi} = \nabla^2\Phi - U' + S,
\eea
so that nothing on the RHS has time derivatives. We will write everything in 2D to reduce the number of indices needed to write the expressions down: the code is in 3D, and we shall explicitly point out any subtleties associates with going from 2D to 3D. 

We   define an ``equation of motion'' term
\bea
\mathcal{E}^t_{i,j} \defn \nabla^2\Phi^t_{i,j} - U'^t_{i,j} + S^t_{i,j},
\eea
where the Laplacian is discretized as
\bea
\nabla^2\Phi^t_{i,j} = \frac{\Phi^t_{i+1,j} + \Phi^t_{i-1,j} - 2 \Phi^t_{i,j}}{h^2} + \frac{\Phi^t_{i,j+1} + \Phi^t_{i,j-1} - 2 \Phi^t_{i,j}}{h^2}.
\eea
The second time derivative at a given location is discretized to second order as
\bea
\ddot{\Phi} = \frac{\Phi^{t+1}_{i,j} + \Phi^{t-1}_{i,j} - 2 \Phi^t_{i,j}}{h_t^2}.
\eea
Hence,
\bea
\Phi_{i,j}^{t+1} = h_t^2\mathcal{E}^t_{i,j}  - \Phi^{t-1}_{i,j} + 2 \Phi^t_{i,j}.
\eea
This gives a rule to update the value of the field at each location.
\subsection{Tests of the code}
To check the code, we have implemented a few different types of initial conditions  and potentials (this also helps to build intuition). To make sure that the code accesses arrays correctly, we begin by solving the gradient flow equation
\bea
\dot{\Phi} = \nabla^2\Phi - U'(\Phi),
\eea
for two different potentials:
\bea
U_{(1)}(\Phi) = \half \Phi^2,\qquad U_{(2)} (\Phi) = \frac{1}{4}(\Phi^2 - 1)^2.
\eea
For a system endowed with the potential $U_{(1)}$, any homogeneous initial field configuration will evolve towards $\Phi =0$, since those are the values which minimise the potential. Similarly, a system endowed with $U_{(2)}$ with homogeneous initial conditions will evolve towards $\Phi = \pm 1$.

The space- and time-step sizes must be carefully chosen: not all values give numerically stable evolutions. For the free wave equation, $\ddot{\Phi} - \nabla^2\Phi =0$, one typically requires
\bea
h_t < h,
\eea
and for the gradient flow equation $\dot{\Phi} - \nabla^2\Phi =0$ one typically requires
\bea
h_t < h^2.
\eea
\appendix
\section{Using the code}
\subsection{Klein-Gordon solver}
The code is written in C$^{++}$, and requires {\tt GSL} and {\tt boost} libraries. To check that these are both installed correctly, nagivate to the {\tt KGsolve} root directory, and type {\tt make clean} then {\tt make}. The code should compile without error. The source code is within the directory {\tt src}.

The {\tt field} is stored as a one-dimensional array; the value of the field at an array location is {\tt field.vals[pos]} (usually, {\tt field->vals} needs to be typed outside of the {\tt main.cpp} file). To access the value of the field corresponding to component {\tt c}, at time {\tt t}, and position on the grid {\tt i,j,k}, one should call the function {\tt ind}:

{\tt pos} = {\tt ind(now,c,i,j,k,grid,field)}


\bibliographystyle{JHEP}
\bibliography{refs}
\end{document}